% Options for packages loaded elsewhere
\PassOptionsToPackage{unicode}{hyperref}
\PassOptionsToPackage{hyphens}{url}
\PassOptionsToPackage{dvipsnames,svgnames,x11names}{xcolor}
%
\documentclass[
  letterpaper,
  DIV=11,
  numbers=noendperiod]{scrartcl}

\usepackage{amsmath,amssymb}
\usepackage{iftex}
\ifPDFTeX
  \usepackage[T1]{fontenc}
  \usepackage[utf8]{inputenc}
  \usepackage{textcomp} % provide euro and other symbols
\else % if luatex or xetex
  \usepackage{unicode-math}
  \defaultfontfeatures{Scale=MatchLowercase}
  \defaultfontfeatures[\rmfamily]{Ligatures=TeX,Scale=1}
\fi
\usepackage{lmodern}
\ifPDFTeX\else  
    % xetex/luatex font selection
\fi
% Use upquote if available, for straight quotes in verbatim environments
\IfFileExists{upquote.sty}{\usepackage{upquote}}{}
\IfFileExists{microtype.sty}{% use microtype if available
  \usepackage[]{microtype}
  \UseMicrotypeSet[protrusion]{basicmath} % disable protrusion for tt fonts
}{}
\makeatletter
\@ifundefined{KOMAClassName}{% if non-KOMA class
  \IfFileExists{parskip.sty}{%
    \usepackage{parskip}
  }{% else
    \setlength{\parindent}{0pt}
    \setlength{\parskip}{6pt plus 2pt minus 1pt}}
}{% if KOMA class
  \KOMAoptions{parskip=half}}
\makeatother
\usepackage{xcolor}
\setlength{\emergencystretch}{3em} % prevent overfull lines
\setcounter{secnumdepth}{-\maxdimen} % remove section numbering
% Make \paragraph and \subparagraph free-standing
\ifx\paragraph\undefined\else
  \let\oldparagraph\paragraph
  \renewcommand{\paragraph}[1]{\oldparagraph{#1}\mbox{}}
\fi
\ifx\subparagraph\undefined\else
  \let\oldsubparagraph\subparagraph
  \renewcommand{\subparagraph}[1]{\oldsubparagraph{#1}\mbox{}}
\fi


\providecommand{\tightlist}{%
  \setlength{\itemsep}{0pt}\setlength{\parskip}{0pt}}\usepackage{longtable,booktabs,array}
\usepackage{calc} % for calculating minipage widths
% Correct order of tables after \paragraph or \subparagraph
\usepackage{etoolbox}
\makeatletter
\patchcmd\longtable{\par}{\if@noskipsec\mbox{}\fi\par}{}{}
\makeatother
% Allow footnotes in longtable head/foot
\IfFileExists{footnotehyper.sty}{\usepackage{footnotehyper}}{\usepackage{footnote}}
\makesavenoteenv{longtable}
\usepackage{graphicx}
\makeatletter
\def\maxwidth{\ifdim\Gin@nat@width>\linewidth\linewidth\else\Gin@nat@width\fi}
\def\maxheight{\ifdim\Gin@nat@height>\textheight\textheight\else\Gin@nat@height\fi}
\makeatother
% Scale images if necessary, so that they will not overflow the page
% margins by default, and it is still possible to overwrite the defaults
% using explicit options in \includegraphics[width, height, ...]{}
\setkeys{Gin}{width=\maxwidth,height=\maxheight,keepaspectratio}
% Set default figure placement to htbp
\makeatletter
\def\fps@figure{htbp}
\makeatother

\KOMAoption{captions}{tableheading}
\makeatletter
\@ifpackageloaded{caption}{}{\usepackage{caption}}
\AtBeginDocument{%
\ifdefined\contentsname
  \renewcommand*\contentsname{Table of contents}
\else
  \newcommand\contentsname{Table of contents}
\fi
\ifdefined\listfigurename
  \renewcommand*\listfigurename{List of Figures}
\else
  \newcommand\listfigurename{List of Figures}
\fi
\ifdefined\listtablename
  \renewcommand*\listtablename{List of Tables}
\else
  \newcommand\listtablename{List of Tables}
\fi
\ifdefined\figurename
  \renewcommand*\figurename{Figure}
\else
  \newcommand\figurename{Figure}
\fi
\ifdefined\tablename
  \renewcommand*\tablename{Table}
\else
  \newcommand\tablename{Table}
\fi
}
\@ifpackageloaded{float}{}{\usepackage{float}}
\floatstyle{ruled}
\@ifundefined{c@chapter}{\newfloat{codelisting}{h}{lop}}{\newfloat{codelisting}{h}{lop}[chapter]}
\floatname{codelisting}{Listing}
\newcommand*\listoflistings{\listof{codelisting}{List of Listings}}
\makeatother
\makeatletter
\makeatother
\makeatletter
\@ifpackageloaded{caption}{}{\usepackage{caption}}
\@ifpackageloaded{subcaption}{}{\usepackage{subcaption}}
\makeatother
\ifLuaTeX
  \usepackage{selnolig}  % disable illegal ligatures
\fi
\IfFileExists{bookmark.sty}{\usepackage{bookmark}}{\usepackage{hyperref}}
\IfFileExists{xurl.sty}{\usepackage{xurl}}{} % add URL line breaks if available
\urlstyle{same} % disable monospaced font for URLs
\hypersetup{
  pdftitle={Response\_to\_Reviews},
  colorlinks=true,
  linkcolor={blue},
  filecolor={Maroon},
  citecolor={Blue},
  urlcolor={Blue},
  pdfcreator={LaTeX via pandoc}}

\title{Response\_to\_Reviews}
\author{}
\date{}

\begin{document}
\maketitle
\textbf{Reviewer comment}

Mash: A lot of the results interpretation is based on interpreting the
SNP loadings on the specified covariance matrices. This is a secondary
use of mash (the primary being the refinement of effect sizes), and
while they do interpret these somewhat in the mash paper, these loadings
need to be interpreted with caution. If these covariance matrices are
similar, mash will somewhat arbitrarily assign weight to any of the
matrices because all lead to the same fit to the effect-size data. The
mash algorithm doesn\textquotesingle t give the full posterior
distribution on the loadings so you can\textquotesingle t check for
posterior correlations there. It seems likely to me that the
hypothesis-driven and data-driven covariance matrices are somewhat
correlated here, and the correlations may differ between green-up and
flowering because of the better correlation between the environmental
metrics and flowering than green-up. In the mash paper, they used
cross-validation with the likelihood in the testing set as the
evaluation metric to compare models. It might be safer to try dropping
specific covariance matrices and comparing the model performance in a
held-out testing set to evaluate the importance of the
hypothesis-derived covariance matrices.

\hfill\break

\textbf{Response}

We thank the reviewer for this comment and agree that the selection of
covariance matrices to be included in a mash model is a nontrivial
question. To clarify our interpretation of the posterior matrix weights
provided by mash, we performed an extensive analysis of the performance
of mash models when different covariance matrices are included.
Specifically, we implemented a model selection approach that uses a
greedy algorithm to evaluate the log likelihood of the mash model as
additional covariance matrices were included (see Methods lines XXX-YYY
and pseudo code below). This is a similar, but more computationally
efficient design, than the leave-one-out approach recommended by the
reviewer as the runtime of mash increases with the number of covariance
matrices included. Additionally, while cross-validation is a powerful
approach to model evaluation our limited sample size was prohibitive to
the necessary partitioning of individuals included in our analysis.

\hfill\break

In practice this greedy algorithm approach offers a fast way to identify
the point where redundancy in the addition of a new covariance matrix
results in no change to the likelihood of the mash model. Importantly,
this removes the potential for arbitrary assignment of weights to the
covariance matrices in the model and allows for the selection of
matrices that most accurately capture the underlying biology captured by
the site-specific effect sizes. We applied this algorithm to only the
hypothesis matrices for each phenotype (Figures 1 and 2) using the
previously used sets of significantly associated and randomly selected
variants in the Gulf subpopulation individuals, Midwest subpopulation
individuals, and the two subpopulations combined. The results indicate
that only a small number of the original hypothesis matrices (between
three and five) are necessary to reach the maximum likelihood model when
using the same likelihood ratio test implemented in Urbut et al.,
indicating that correlation among hypothesis covariance matrices is
high. This redundancy was confirmed through our application of the
greedy algorithm to a combined set of canonical, data driven, and
hypothesis covariance matrices (Figures 3 and 4)

Our primary conclusion from this analysis is in line with what the
reviewer posited, that there is extensive redundancy among canonical,
data driven, and hypothesis covariance matrices that is leading to
biologically informative patterns of covariance between site effect
sizes to be arbitrarily distributed among them. We additionally found
evidence that while the correlations between data driven and
environmentally informed (hypothesis) covariance matrices, there is
extensive similarity in the matrices that are included in the maximum
likelihood models across both phenotypes and in the analysis of the Gulf
subpopulation, Midwest subpopulations, and the combined cohort.

~

\_\_\_\_\_\_\_\_\_\_\_\_\_\_\_\_\_\_\_\_\_\_\_\_\_\_\_\_\_\_\_\_\_\_\_\_\_\_\_\_\_\_\_\_\_\_\_\_\_\_\_\_\_\_\_\_\_\_\_\_\_\_\_\_\_\_\_\_\_\_\_\_\_\_\_\_

Greedy algorithm for mash model selection.~

\_\_\_\_\_\_\_\_\_\_\_\_\_\_\_\_\_\_\_\_\_\_\_\_\_\_\_\_\_\_\_\_\_\_\_\_\_\_\_\_\_\_\_\_\_\_\_\_\_\_\_\_\_\_\_\_\_\_\_\_\_\_\_\_\_\_\_\_\_\_\_\_\_\_\_\_

\hfill\break

matrices = {[}all covariance matrices of interest{]}

most\_likely\_matrices = {[}{]} \#a list to store the most likely
additional matrix at each iteration

maximum\_likelihoods = {[}{]}\#a list to store the maximum likelihood at
each iteration

\hfill\break

likelihood = -inf

most\_likely\_matrix = \textquotesingle\textquotesingle{}

for matrix in matrices:

\begin{verbatim}
matrix_likelihood = mash(effects, std.errs, matrix)\[\'likelihood\'\] #obtain model likelihood

if matrix_likelihood \> likelihood:

    most_likely_matrix = matrix

    likelihood = matrix_likelihood
\end{verbatim}

\hfill\break

matrices.remove(most\_likely\_matrix)

most\_likely\_matrices.append(most\_likely\_matrix)

maximum\_likelihoods.append(likelihood)

\hfill\break

lrt\_pvalue = 0

iteration = 2

while lrt\_pvalue \textless{} 0.05:

\begin{verbatim}
likelihood = -inf
\end{verbatim}

most\_likely\_matrix = \textquotesingle\textquotesingle{}

\begin{verbatim}
for matrix in matrices:

    model_matrices = most_likely_matrices + matrix 

    model_likelihood = mash(effects, std.errs, model_matrices)\[\'likelihood\'\]

if matrix_likelihood \> likelihood:

    most_likely_matrix = matrix

    likelihood = matrix_likelihood
\end{verbatim}

\hfill\break

matrices.remove(most\_likely\_matrix)

most\_likely\_matrices.append(most\_likely\_matrix)

maximum\_likelihoods.append(likelihood)

\hfill\break

\#perform a likelihood ratio test of the most likely model from the
previous iteration to the model in this step with one degree of freedom

lrt\_pvalue = likelihood\_ratio\_test(likelihood,
maximum\_likelihoods{[}iteration-1{]}, df = 1)

\_\_\_\_\_\_\_\_\_\_\_\_\_\_\_\_\_\_\_\_\_\_\_\_\_\_\_\_\_\_\_\_\_\_\_\_\_\_\_\_\_\_\_\_\_\_\_\_\_\_\_\_\_\_\_\_\_\_\_\_\_\_\_\_\_\_\_\_\_\_\_\_\_\_\_\_

\_\_\_\_\_\_\_\_\_\_\_\_\_\_\_\_\_\_\_\_\_\_\_\_\_\_\_\_\_\_\_\_\_\_\_\_\_\_\_\_\_\_\_\_\_\_\_\_\_\_\_\_\_\_\_\_\_\_\_\_\_\_\_\_\_\_\_\_\_\_\_\_\_\_\_\_

\hfill\break

\textbf{Methods}

In order to better understand the behavior of the mash shrinkage
algorithm in the presence of correlated covariance matrices, we
implemented a "greedy" mash algorithm. Specifically, given n covariance
matrices for each phenotype and corresponding effect sizes and standard
errors across the k contexts of interest, we fit a set of mash models
with each of the n matrices separately. We then identify the model with
the maximum log likelihood estimates and select the corresponding
matrix, leaving n-1 covariance matrices. We then fit a new set of mash
models where the selected covariance matrix is paired with each of the
remaining n-1 covariance matrices. The most likely pair of covariance
matrices are then selected and the process is repeated for all possible
n-2 matrix triplets. This process is repeated until one of two stop
conditions are met: (i) all covariance matrices have been added to the
mash model, resulting in a final model with all of the original n
covariance matrices are included or (ii) a likelihood ratio test with
one degree of freedom between the likelihood of the current model and
the model from the previous iteration has a p-value \textgreater{} 0.05.
The result is a set of log likelihood estimates for each model and the
stepwise \textquotesingle path\textquotesingle{} of covariance matrices
that best explain the observed effect sizes and standard errors observed
for the phenotype of interest.~

\hfill\break

We then applied the greedy algorithm to two sets of covariance matrices.
First, we applied it to only the hypothesized covariance matrices for
each phenotype as described in Supplementary Table 1. Briefly, the
hypothesis covariance matrices included in our analysis of greenup time
included: average temperature in the 5 days, 10 days, and 18 days
preceding greenup and the number of cumulative growing degree days (over
12 degrees Celsius) in the 5 days, 10 days, and 18 days preceding
greenup. Each of these six covariance matrices were calculated using
individuals who were present at both sites. Additionally, each matrix
was calculated using all individuals, only individuals from the Gulf
subpopulation, and only individuals from the Midwest subpopulation.

\hfill\break

For the flowering time phenotype, 12 hypothesized covariance matrices
were included in the greedy algorithm. For each phenotype, we used the
mash greedy algorithm to identify the most likely covariance matrix at
each step using both a set of random independent variants and a set of
independent variants that showed the strongest associations with the
phenotype across all contexts. The resulting maximum likelihood paths
for the selection of covariance matrices as determined by the greedy
algorithm for each phenotype, subpopulation, and set of randomly
selected or significantly associated variants are shown in Figures 1 and
2.~

\hfill\break

Our second analyses expanded the set of covariance matrices to all of
the matrices included in our analyses of the empirical data. These
include: the phenotype specific covariance matrices described above, the
data-driven covariance matrices, and the suite of canonical covariance
matrices representing context-specific effects, equal effects across
contexts, and simple scenarios of heterogeneous effects across contexts.
The resulting maximum likelihood paths for the selection of covariance
matrices as determined by the greedy algorithm for each phenotype,
subpopulation, and set of randomly selected or significantly associated
variants are shown in Figures 3 and 4.~

\hfill\break

\textbf{Results}

We first applied the greedy algorithm to only the set of hypothesis
matrices for each of the greenup and flowering phenotypes, respectively.
Generally, we observed that only a small number of matrices
significantly increased the likelihood of the model as the algorithm
progressed. In our analysis of the set of variants that were
significantly associated with greenup time in at least one context, 18
hypothesis covariance matrices were included in the initialization of
the greedy algorithm. The stop condition, a likelihood ratio test
p-value \textgreater{} 0.05 between the most likely model at step n and
most likely model at step n+1, was met after only four iterations in
both the combined cohort (Figure 1A) as well as the analysis of the Gulf
subpopulation alone (Figure 1B). In our analysis of the midwest
subpopulation the stop condition was met after only three iterations,
see Figure 1C. The relatively small number of necessary iterations, and
correspondingly small number of covariance matrices, indicate that the
hypothesis covariance matrices were highly correlated with one another
and that inclusion of all the hypothesis covariance matrices introduced
redundancy in the model. In the set of randomly selected independent
variants, the stop condition was met in the combined cohort after five
steps (Figure 1D). The stop condition was met after four and three
iterations in the Gulf and Midwest subpopulations, respectively. The
overlap between matrices included in the maximum likelihood model prior
to reaching the stop condition in the analysis of significantly
associated and randomly selected variants was high in the combined
cohort; three of four matrices in the analysis of significantly
associated variants were also included in the model for the randomly
selected variants. Similarly, three of four matrices in the Gulf
subpopulation and two of three matrices in the Midwest subpopulation
that were included in the model for significantly associated variants
were also included in the model for the randomly selected variants.~

\hfill\break

We then applied the greedy algorithm to the 12 hypothesis covariance
matrices for the flowering phenotype. Similar to our analysis of the
hypothesis covariance matrices for the greenup phenotype, we observed a
similar pattern of correlation in hypothesis covariance matrices for the
flowering phenotype. The stop condition for the greedy algorithm was met
after five steps when analyzing the combined cohort, the Gulf
subpopulation, and the Midwest subpopulation in analysis of variants
that were significantly associated in at least one context (Figure
2A-C). When analyzing the set of randomly selected variants the stop
condition was met after four steps in the combined cohort, five steps in
the Gulf subpopulation, and six steps in the Midwest subpopulation. We
also observed that many of the matrices included in the model for the
significantly associated variants were also included in the
corresponding model for randomly selected variants (Figure 2D-F).~

\hfill\break

Analysis of the hypothesis covariance matrices alone helped to establish
the utility of the greedy algorithm implementation in identifying the
maximum likelihood mash model for the observed summary statistics from
our analysis of eight sites. However, due to substantial evidence that
both canonical and data driven matrices explained a nontrivial
proportion of the observed covariance between context specific summary
statistics we applied the greedy algorithm with the all three classes of
covariance matrices included in the initial set. In our analysis of the
greenup phenotype, the stop condition was met after 16 iterations,
resulting in a maximum likelihood model that included all 13 canonical
covariance matrices, two data driven matrices (total PCA and PC 4), and
the hypothesis matrix representing the covariance in average temperature
ten days prior to greenup in the combined cohort (Figure 3A). When the
Gulf subpopulation was analyzed alone the stop condition was met after
five iterations. The five corresponding matrices were composed of four
canonical matrices and the hypothesis matrix calculated as the
covariance in average temperature five days prior to greenup in only
individuals from the Gulf subpopulation (Figure 3B). Finally, the greedy
algorithm met the stop condition after 13 iterations when applied to the
Midwest subpopulation. Ten of the canonical covariance matrices, the
total PCA data driven covariance matrix, and the two covariance matrices
calculated using the five day temperature average in the five days prior
to greenup and the number of growing degree days in the 18 days prior to
greenup in the Midwest subpopulation individuals were all include in the
mode (Figure 3C). Most notably, the canonical covariance matrix
representing independent effects was the first matrix selected in all
three applications of the greedy algorithm, indicating a large number of
independent effects were present in our analysis of the greenup
phenotype. We then applied the greedy algorithm to a set of randomly
selected variants and observed that many of the same matrices included
in the corresponding models from application to the significantly
associated variants were observed in the most likely models once the
stop condition was met (Figure 3D-F).~

\hfill\break

We then performed the same analysis of the entire suite of canonical,
data driven, and hypothesis covariance matrices for the flowering
phenotype. The stop condition was met after five iterations in the
combined cohort and included four canonical covariance matrices
representing independent effects, equal effects, and two simple models
of effect heterogeneity among sites (Figure 4A). In application to the
Gulf subpopulation, the stop condition was met after 11 iterations and
included 10 canonical covariance matrices and the data driven covariance
matrix derived from the third principal component (Figure 4B). The stop
condition was met after 11 iterations in the analysis of Midwest
subpopulation individuals and included nine canonical covariance
matrices, the data driven matrix derived from the first principal
component, and the hypothesis matrix calculated using the cumulative
rainfall in the day prior to flowering tin the Midwest subpopulation
(Figure 4C). Once again, we observed that the first matrix selected in
each analysis of flowering time was the canonical matrix representing
independent effects among conditions. Overlap between the models
selected by the greedy algorithm when analyzing a set of randomly
selected variants was high across all three analyses (Figure 4C-D).

\hfill\break

\textbf{Figures}

\hfill\break
\hfill\break

\includegraphics{response_to_reviews_files/mediabag/ae_Fi1djzcYo_ibeQWWj.pdf}

\textbf{Figure 1. Log-likelihood at each step in a greedy algorithm
implementation of model selection for mash for the greenup phenotype
using only hypothesis covariance matrices.} The matrix that resulted in
the maximum likelihood model of covariance in summary statistics among
sites is located at the base of the y-axis; matrices selected in the
ensuing step are placed in ascending order along the y-axis.
\textbf{(A)} The maximum likelihood path of covariance matrices in the
greedy algorithm for the set of independent variants that were
significant in at least one context when analyzing all individuals. The
dashed black line represents the step in the algorithm where addition of
the next matrix did not significantly increase the model likelihood,
likelihood-ratio test p-value \textgreater{} 0.05. \textbf{(B)} and
\textbf{(C)} show the maximum likelihood paths when the individuals
analyzed are only from the Gulf and Midwest subpopulations,
respectively. \textbf{(D)} The maximum likelihood path of covariance
matrices in the greedy algorithm for the set of independent, randomly
selected variants when analyzing all individuals. The dashed black line
represents the step in the algorithm where addition of the next matrix
did not significantly increase the model likelihood, likelihood-ratio
test p-value \textgreater{} 0.05. * is used to denote those matrices
that are identified by the greedy algorithm in the set of significantly
associated variants and the randomly associated variants. \textbf{(E)}
and \textbf{(F)} show the maximum likelihood paths when the individuals
analyzed are only from the Gulf and Midwest subpopulations,
respectively.

\hfill\break

\includegraphics{response_to_reviews_files/mediabag/8AuFUNJTaIAgX7GVkOSn.pdf}

\textbf{Figure 2. Log-likelihood at each step in a greedy algorithm
implementation of model selection for mash for the flowering phenotype
using only hypothesis covariance matrices.} The matrix that resulted in
the maximum likelihood model among sites is located at the base of the
y-axis; matrices selected in the ensuing step are placed in ascending
order along the y-axis. \textbf{(A)} The maximum likelihood path of
covariance matrices in the greedy algorithm for the set of independent
variants that were significant in at least one context when analyzing
all individuals. The dashed black line represents the step in the
algorithm where addition of the next matrix did not significantly
increase the model likelihood, likelihood-ratio test p-value
\textgreater{} 0.05. \textbf{(B)} and \textbf{(C)} show the maximum
likelihood paths when the individuals analyzed are only from the Gulf
and Midwest subpopulations, respectively. \textbf{(D)} The maximum
likelihood path of covariance matrices in the greedy algorithm for the
set of independent, randomly selected variants when analyzing all
individuals. The dashed black line represents the step in the algorithm
where addition of the next matrix did not significantly increase the
model likelihood, likelihood-ratio test p-value \textgreater{} 0.05. *
is used to denote those matrices that are identified by the greedy
algorithm in the set of significantly associated variants and the
randomly associated variants. \textbf{(E)} and \textbf{(F)} show the
maximum likelihood paths when the individuals analyzed are only from the
Gulf and Midwest subpopulations, respectively.

\hfill\break

\includegraphics{response_to_reviews_files/mediabag/iO_JwhiNJbDDdV-UHqzl.pdf}

\textbf{Figure 3. Log-likelihood at each step in a greedy algorithm
implementation of model selection for mash for the greenup phenotype
using canonical, data drive, and hypothesis covariance matrices.} The
matrix that resulted in the maximum likelihood model of covariance in
summary statistics among sites is located at the base of the y-axis;
matrices selected in the ensuing step are placed in ascending order
along the y-axis. \textbf{(A)} The maximum likelihood path of covariance
matrices in the greedy algorithm for the set of independent variants
that were significant in at least one context when analyzing all
individuals. The dashed black line represents the step in the algorithm
where addition of the next matrix did not significantly increase the
model likelihood, likelihood-ratio test p-value \textgreater{} 0.05.
Once this condition was met the greedy algorithm was halted.
\textbf{(B)} and \textbf{(C)} show the maximum likelihood paths when the
individuals analyzed are only from the Gulf and Midwest subpopulations,
respectively. \textbf{(D)} The maximum likelihood path of covariance
matrices in the greedy algorithm for the set of independent, randomly
selected variants when analyzing all individuals. The dashed black line
represents the step in the algorithm where addition of the next matrix
did not significantly increase the model likelihood, likelihood-ratio
test p-value \textgreater{} 0.05. Once this condition was met the greedy
algorithm was halted for computational considerations. * is used to
denote those matrices that are identified by the greedy algorithm in the
set of significantly associated variants and the randomly associated
variants. \textbf{(E)} and \textbf{(F)} show the maximum likelihood
paths when the individuals analyzed are only from the Gulf and Midwest
subpopulations, respectively.

\hfill\break

\includegraphics{response_to_reviews_files/mediabag/LnnT8NIjtolmYO5neLCa.pdf}

\textbf{Figure 4. Log-likelihood at each step in a greedy algorithm
implementation of model selection for mash for the flowering phenotype
using canonical, data drive, and hypothesis covariance matrices.} The
matrix that resulted in the maximum likelihood model of covariance in
summary statistics among sites is located at the base of the y-axis;
matrices selected in the ensuing step are placed in ascending order
along the y-axis. \textbf{(A)} The maximum likelihood path of covariance
matrices in the greedy algorithm for the set of independent variants
that were significant in at least one context when analyzing all
individuals. The dashed black line represents the step in the algorithm
where addition of the next matrix did not significantly increase the
model likelihood, likelihood-ratio test p-value \textgreater{} 0.05.
Once this condition was met the greedy algorithm was halted.
\textbf{(B)} and \textbf{(C)} show the maximum likelihood paths when the
individuals analyzed are only from the Gulf and Midwest subpopulations,
respectively. \textbf{(D)} The maximum likelihood path of covariance
matrices in the greedy algorithm for the set of independent, randomly
selected variants when analyzing all individuals. The dashed black line
represents the step in the algorithm where addition of the next matrix
did not significantly increase the model likelihood, likelihood-ratio
test p-value \textgreater{} 0.05. Once this condition was met the greedy
algorithm was halted for computational considerations. * is used to
denote those matrices that are identified by the greedy algorithm in the
set of significantly associated variants and the randomly associated
variants. \textbf{(E)} and \textbf{(F)} show the maximum likelihood
paths when the individuals analyzed are only from the Gulf and Midwest
subpopulations, respectively.



\end{document}
